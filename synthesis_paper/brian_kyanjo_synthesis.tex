\documentclass[12pt,a4paper]{article}

%%%%%%%%%%%%%%%%%%%%%%%%% packages %%%%%%%%%%%%%%%%%%%%%%%%
\usepackage{amsmath}
\usepackage{amssymb}
\usepackage{amsthm}
\usepackage{amsfonts}
\usepackage{graphicx}
\usepackage[utf8]{inputenc}
\usepackage[english]{babel}
\usepackage[all]{xy}
\usepackage{float}
\usepackage{tikz}
\usepackage{verbatim}
\usepackage[left=2cm,right=2cm,top=2cm,bottom=2cm]{geometry}
\usepackage{hyperref}
\usepackage{caption}
\usepackage{subcaption}
\usepackage{psfrag}
\usepackage{natbib}
%\bibliographystyle{abbrvnat}
\usepackage{booktabs}  
\usepackage[T1]{fontenc}    % use 8-bit T1 fonts
\usepackage{url} 


%%%%%%%%%%%%%%%%%%%%% students data %%%%%%%%%%%%%%%%%%%%%%%%
\newcommand{\student}{Brian KYANJO }
\newcommand{\course}{Prof. Jodi Mead and Prof. Dylan Mikesell}
\newcommand{\assignment}{ Prof. Donna Calhoun}

%%%%%%%%%%%%%%%%%%% using theorem style %%%%%%%%%%%%%%%%%%%%
\newtheorem{thm}{Theorem}
\newtheorem{lem}[thm]{Lemma}
\newtheorem{defn}[thm]{Definition}
\newtheorem{definition}{Definition}[section] 
\newtheorem{theorem}{Theorem}
\newtheorem{exa}[thm]{Example}
\newtheorem{rem}[thm]{Remark}
\newtheorem{coro}[thm]{Corollary}
\newtheorem{quest}{Question}[section]

%%%%%%%%%%%%%%  Shortcut for usual set of numbers  %%%%%%%%%%%

\newcommand{\N}{\mathbb{N}}
\newcommand{\Z}{\mathbb{Z}}
\newcommand{\Q}{\mathbb{Q}}
\newcommand{\R}{\mathbb{R}}
\newcommand{\C}{\mathbb{C}}

%%%%%%%%%%%%%%%%%%%%%%%%%%%%%%%%%%%%%%%%%%%%%%%%%%%%%%%555
\begin{document}
	
	%%%%%%%%%%%%%%%%%%%%%%% title page %%%%%%%%%%%%%%%%%%%%%%%%%%
	\thispagestyle{empty}
	\begin{center}
		\textbf{Reimann Solver in Geoclaw \\[0.5cm]
		Synthesis Paper}
		\vspace{.2cm}
	\end{center}
	
	%%%%%%%%%%%%%%%%%%%%% assignment information %%%%%%%%%%%%%%%%
	\noindent
	\rule{17cm}{0.2cm}\\[0.3cm]
	Name: \student \hfill Supervisor: \assignment\\[0.1cm]
	Committee: \course \hfill Date: \today\\
	\rule{17cm}{0.05cm}
	\vspace{.2cm}
	
	\section{Introduction}
	

	\section{Wave Propagation Algorithm (WPA)}
    
    {\bf Explain what a finite scheme is.  }
    
    The  order one dimensional (1D)  wave propagation method is given by eqaution \eqref{wpa1}
	
	{\bf Explain where $\mathcal{A^{+}}\Delta Q_{i-\frac{1}{2}}^{n}$  and $\mathcal{A^{-}}\Delta Q_{i-\frac{1}{2}}^{n}comes from}
	
	\begin{equation}
		Q_{i}^{n+1} =  Q_{i}^{n} - \frac{\Delta t}{\Delta x}(\mathcal{A^{+}}\Delta 	Q_{i-\frac{1}{2}}^{n} + \mathcal{A^{-}}Q_{i+\frac{1}{2}}^{n})
		\label{wpa1}
	\end{equation}
	
	% Avoid putting integral symbols, and so on, "inline". Put them in their own equation environemnt. 
	\noindent where  $Q_{i}^{n}$ is a numerical approximation to $\dfrac{1}{\Delta x} \int_{C_{i}}q(x,t^{n})dx$, $\Delta x = (x_{i+\frac{1}{2}} - x_{i-\frac{1}{2}})$, $\Delta t = (t^{n+1} - t^{n})$, $\mathcal{A^{\pm}}\Delta 	Q_{i\pm\frac{1}{2}}^{n} $ are fluctuations determined by the to the Riemann Problems at cell interfaces at $x_{i\pm\frac{1}{2}}$. The net updating contributions from the rightward and leftward moving waves into the grid cell $C_{i}$ from the right and left interface are respectively given by $\mathcal{A^{+}}\Delta 	Q_{i-\frac{1}{2}}^{n}$ and  $\mathcal{A^{-}}Q_{i+\frac{1}{2}}^{n}$\cite{ge:2008}.
	
	The second order accuracy is obtained by taking the correction terms into account as shown described in equation \eqref{wpa2}
	
	\begin{equation}
		Q_{i}^{n+1} =  Q_{i}^{n} - \frac{\Delta t}{\Delta x}(\mathcal{A^{+}}\Delta 	Q_{i-\frac{1}{2}}^{n} + \mathcal{A^{-}}Q_{i+\frac{1}{2}}^{n}) -  \frac{\Delta t}{\Delta x} (\tilde{F}_{i+\frac{1}{2}}^{n} - \tilde{F}_{i-\frac{1}{2}}^{n} )
		\label{wpa2}
	\end{equation}
	\noindent where $\tilde{F}_{i\pm\frac{1}{2}}^{n} $ are second order correction terms determined the waves in the Riemann problems at $x_{i\pm \frac{1}{2}}$.
	
	
	\section{Shallow Water Equations (SWE)}
	
The 1D SWE are given in equations \eqref{swe1} and \eqref{swe2} below.
% Use the eqnarray environment for both equations
 	\begin{equation}	
		h_{t} + (hu)_{x} = 0
		\label{swe1}
	\end{equation}

\begin{equation}
	(hu)_{t} + \left(hu^{2} + \frac{1}{2}gh^{2} \right)_{x} = -ghb_{x}
	\label{swe2}
\end{equation}
where $h(x,t)$ is the fluid depth, $u(x,t)$ is the vertically averaged horizontal fluid velocity, $g$ is the gravitataional constant, $u(x,t)$ is the vertically averaged horizontal fluid velocity, and $b(x)$ is the bottom surface elevation \cite{ge:2008}.
	\section{Reimann Problem for Wet/Dry States}
	
		\section{Numerical Examples}

	\bibliographystyle{plain}
	\bibliography{geoclaw}
	
\end{document}
